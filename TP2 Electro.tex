\documentclass[a4paper]{article}
\usepackage[utf8]{inputenc}
\usepackage[spanish, es-tabla]{babel}

\usepackage{amsmath}
\usepackage{amsfonts}
\usepackage{amssymb}

\usepackage{float}
\usepackage{graphicx}
\graphicspath{ {./Imagenes/} }

\usepackage[american voltage]{circuitikz}

\usepackage{fancyhdr}

\usepackage{units} 

\pagestyle{fancy}
\fancyhf{}
\lhead{93.54 Métodos Numéricos}
\rhead{Fontecha, Lambertucci, Pouthier, Londero B.}
\rfoot{Página \thepage}



\begin{document}

%%%%%%%%%%%%%%%%%%%%%%%%%%%%%%%%%%%%%%%%%%%%%%%%%%%%%%%%%%%%%%%%%%%%%%%%% 
%								CARATULA								%
%%%%%%%%%%%%%%%%%%%%%%%%%%%%%%%%%%%%%%%%%%%%%%%%%%%%%%%%%%%%%%%%%%%%%%%%% 

\begin{titlepage}
\newcommand{\HRule}{\rule{\linewidth}{0.5mm}}
\center
\mbox{\textsc{\LARGE \bfseries {Instituto Tecnológico de Buenos Aires}}}\\[1.5cm]
\textsc{\Large 22.02 Electrotecnia I}\\[0.5cm]


\HRule \\[0.6cm]
{ \Huge \bfseries Trabajo práctico N$^{\circ}$2}\\[0.4cm] 
\HRule \\[1.5cm]


{\large

\emph{Grupo 5}\\
\vspace{3px}

\begin{tabular}{lr} 	
\textsc{Mechoulam}, Alan  &  \\
\textsc{Lambertucci}, Guido Enrique  & 58009 \\
\textsc{Pouthier}, Florian  & 61337 \\
\textsc{Mestanza}, Nicolás  & 61337 \\
\textsc{Londero Bonaparte}, Tomás Guillermo  & 58150 \\
\end{tabular}

\vspace{20px}

\emph{Profesores}\\
\vspace{3px}
\textsc{Muñoz}, Claudio Marcelo\\ 	
\textsc{Ayub}, Gustavo\\ 	

\vspace{100px}

\begin{tabular}{ll}

Presentado: & 26/04/19\\

\end{tabular}

}

\vfill

\end{titlepage}


%%%%%%%%%%%%%%%%%%%%%%%%%%%%%%%%%%%%%%%%%%%%%%%%%%%%%%%%%%%%%%%%%%%%%%%%% 
%								INFORME									%
%%%%%%%%%%%%%%%%%%%%%%%%%%%%%%%%%%%%%%%%%%%%%%%%%%%%%%%%%%%%%%%%%%%%%%%%%

\textbf{\underline{Ejercicio 1}}

En ese primer ejercicio, el objetivo fue de determinar la configuración de un circuito  dispuesto en una caja, pudiendo ser RC serie o RC paralelo. Para hallarlo, hicimos diferentes medidas alrededor del circuito.
\begin{itemize}
\item[$\bullet$] 
\item[$\bullet$]
\item[$\bullet$]
\end{itemize}

\begin{figure}[H]
\begin{center}
\begin{circuitikz}
	\draw
	(0,0) 	to (2.5,0)
		 	to [R, l=$R$] (4.5,0) 
			to (8,0)
			to [short, -o](9,0)
	(0,0)	to [V, l=$V_{1}$] (0,-3)
	(5.5,0)	to [C, l=$C$] (5.5,-3)
	(0,-3) 	to (8,-3) 
			to [short, -o](9,-3);
	\draw[dashed]
	(2,-4) to (2,1) to (7,1) to (7,-4) to (2,-4);
\end{circuitikz}
\end{center}
\caption{Circuito RC serie}
\label{RCserie}
\end{figure}

%%%%%%%%%%%%%%%%%%%%%%%%%
%		Pregunta d		%
%%%%%%%%%%%%%%%%%%%%%%%%%
Colocando un óhmetro en paralelo del circuito, se pudo medir una resistencia $R_{exp}=223.5\Omega$.
Fijandose al 

%%%%%%%%%%%%%%%%%%%%%%%%%
%		Pregunta e		%
%%%%%%%%%%%%%%%%%%%%%%%%%

\newpage

\textbf{\underline{Ejercicio 2}}

\vspace{1em}

Se considera ahora el circuito RLC serie de la Figura \ref{RLCserie}.

\begin{figure}[H]
\begin{center}
\begin{circuitikz}
	\draw
	(0,0) 	to [V, l=$V_{2}$] (0,-3)
	(0,0) 	to (1,0) 
	(3,0)	to (2.5,0)
			to [spst] (2,0)
			to (1,0)
	(2,-3)	to (2,-0.5)
	(3,0)	to [L, l=$L$] (5,0)
			to [C, l=$C$] (7,0)
			to [R, l=$R$] (7,-3)
			to (0,-3);
\end{circuitikz}
\end{center}
\caption{Circuito RLC serie}
\label{RLCserie}
\end{figure}

Resolvando ese circuito usando el \textit{método de las mallas}, se puede escribir :
\begin{equation}\label{mallas}
v_{2}(t) = v_{R}(t)+v_{L}(t)+v_{C}(t).
\end{equation}

Sin embargo, sabemos que :
\begin{equation}\label{vryvc}
v_{R}=R \cdot i(t)
\quad\text{y}\quad
v_{L}=L\cdot\frac{di(t)}{dt}
\end{equation}
y también que la corriente en el circuito es dado por :
\begin{equation}\label{corriente}
i(t)=C \cdot \frac{dv_{C}(t)}{dt}
\end{equation}

Reemplazando por la expresión (\ref{corriente}) de la corriente en las expresiones de (\ref{vryvc}), se obtiene :
\begin{equation}
v_{R}(t)=RC\cdot\frac{dv_{C}(t)}{dt}
\quad\text{y}\quad
v_{L}(t)=LC\cdot\frac{d^{2}v_{C}(t)}{dt^{2}}.
\end{equation}

Entonces, la ecuación (\ref{mallas}) se escribe
\begin{equation}
LC\cdot\frac{d^{2}v_{C}(t)}{dt^{2}}+RC\cdot\frac{dv_{C}(t)}{dt}+v_{C}(t)=v_{2}(t),
\end{equation}
o también
\begin{equation}
\frac{d^{2}v_{C}(t)}{dt^{2}}+\frac{R}{L}\cdot\frac{dv_{C}(t)}{dt}+\frac{1}{LC}\cdot v_{C}(t)=\frac{1}{LC}\cdot v_{2}(t).
\end{equation}


En el caso de que $R_{t}=0$, la ecuación del circuito se vuelve en la siguiente :
\begin{equation}
\frac{d^{2}v_{C}(t)}{dt^{2}}+\frac{1}{LC}\cdot v_{C}(t)=\frac{1}{LC}\cdot v_{2}(t).
\end{equation}

La solución de esa ecuación diferencial se escribe :
\begin{equation}
v_{C}(t)=v_{Ch}(t)+v_{Cp}(t),
\end{equation}
donde :
\begin{itemize}
\item[$\bullet$] $v_{Cp}(t)$ es la \emph{solución particular} de la ecuación o modo forzado, dado por :
\begin{equation}
v_{Cp}(t)=V_{2}=v_{2}(t\to\infty)
\end{equation}
\item[$\bullet$] $v_{Ch}(t)$ es la \emph{solución homogéneo} de la ecuación, es decir la solución de la ecuación (\ref{eqHom}) :
\begin{equation}\label{eqHom}
\frac{d^{2}v_{Ch}(t)}{dt^{2}}+\frac{1}{LC}\cdot v_{Ch}(t)=0
\quad\Leftrightarrow\quad
\frac{d^{2}v_{Ch}(t)}{dt^{2}}+\omega_{0}^{2}\cdot v_{Ch}(t)=0
\end{equation}
donde
\begin{equation}
\omega_{0}=\frac{1}{\sqrt{LC}}.
\end{equation} 
Considerando que $v_{Ch}(t)$ es de la forma $v_{Ch}(t)=e^{\beta t}$, se halla sustituando esa expresión en (\ref{eqHom}):
\begin{equation}
\beta^{2}e^{\beta t}+\omega_{0}^{2}e^{\beta t}=0
\quad\Leftrightarrow\quad
\beta^{2}+\omega_{0}^{2}=0
\end{equation}
\end{itemize}

Para concluir, en el caso de $R_{t}=0$, se obtiene un circuito LC que se comporta como un oscilador libre no amortiguado.

\end{document}
